\documentclass[a4paper, 12pt]{article}
\usepackage[utf8]{inputenc}
\usepackage[hmarginratio=1:1,margin=2.5cm]{geometry}
\usepackage{titling}
\usepackage{caption}
\usepackage{subcaption}
\usepackage{hyperref}
\usepackage{graphicx}
\usepackage{siunitx}
\usepackage{wrapfig}
%\usepackage{biblatex}
\usepackage{tabularx}
\usepackage{enumitem}
\usepackage{minted}
\usepackage{bookmark}
\usepackage{setspace}

\hypersetup{
  colorlinks, linkcolor=blue
}

%\setlength{\droptitle}{6cm}
%\setlength\parskip{\bigskipamount}
%\title{An Introduction to the Spacecraft Plasma\\ Interaction Software (SPIS)}
%\date{\today}
%\author{M.K.G. Holmberg, D. van Winden, H. Adamski, N.M. Besch}

\begin{document}
 \begin{titlepage}
    \vspace*{\stretch{0.5}}
    \begin{center}
      {\Large\bfseries An Introduction to the Spacecraft Plasma\\[1ex] Interaction Software (SPIS)}\\[2cm]
      {\large M.K.G. Holmberg, D. van Winden, H. Adamski, N.M. Besch}\\[2cm]
      \today
    \end{center}
    \vspace*{\fill}
  \end{titlepage}
%\maketitle
\voffset=20pt
\setlength{\textheight=660pt}
\setlength{\parindent=0pt}
\newpage
\doublespacing
\tableofcontents
\singlespacing
\newpage

\section{Introduction to spacecraft plasma interaction}

\subsection{The Spacecraft Plasma Interaction Software (SPIS)}
\vspace{2mm}

SPIS is a software used to model the interaction between an object, typically a spacecraft, and its surrounding space environment. The interaction is primarily driven by absorption or emission of charged particles. However, simulating the dynamics and emission or absorption of each particle would require immense computational resources. To address this, SPIS employs the particle-in-cell (PIC) method, which provides an efficient approximation. Within PIC individual particles are tracked in continuous phase space, while the movements of the distribution, e.g., densities and currents, are computed simultaneously on Eulerian mesh points.\\

SPIS is a Java based software. Packages and updates have been added to the software during the last 20 years, which makes it difficult to get a good overview of what the code actually does. The user interface is good enough so that it's possible to use the software without knowledge of the code. This is, however, not advisable since the user most likely will come across problems when running the simulations or unrealistic results, which can only be explained by a good understanding of the code. This report will therefore include both a description of how to run the simulations and short code sections describing what is actually being executed (not yet included in v5 of this guide). This will help the user to dig deeper into the code themselves and to retrieve the knowledge that is needed for their specific project.

\subsection{Download and install}
\vspace{2mm}

The first step in running SPIS simulations is to register as a member of the Spacecraft Plasma Interactions Network in Europe (SPINE), the official community for all SPIS users. The registration is made on the SPINE registration page: \url{https://www.spis.org/register/}. Click the button labeled \verb|SPINE Community login form| and on the following page, below the \verb|Sign In| button, select the option to \verb|Register|. Once the registration is approved, SPIS can be downloaded from \url{https://www.spis.org/get/software/spis/surface/latest/}. This link will open a page similar to the window shown in Figure \ref{1}, where the appropriate version of SPIS can be selected based on the operating system. Please make sure to download the latest version of the software. The instructions presented here refer to SPIS version 6.1.0 as this was the latest version at the time of writing. If choosing \verb|Download| and entering a username and password, the site redirects to the page shown in Figure \ref{2}. By selecting \verb|Download software|, it will redirect again to the page shown in Figure \ref{3}. From there, selecting \verb|spis/|, \verb|surface/|, \verb|latest/| will lead to the same page as shown in Figure \ref{1}.\\

\begin{figure}[!ht]
    \centering
    \includegraphics[width=1\textwidth]{fig1.jpg}
    \caption{The page where SPIS can be downloaded.}
    \label{1}
\end{figure}

\begin{figure}[!ht]
    \centering
    \includegraphics[width=1\textwidth]{fig2.jpg}
    \caption{The webpage where SPIS can be downloaded, following authentication via a registered user account.}
    \label{2}
\end{figure}

\begin{figure}[!ht]
    \centering
    \includegraphics[width=1\textwidth]{fig3.jpg}
    \caption{The page that appears after selecting "Download software".}
    \label{3}
\end{figure}

The minimum hardware requirements to properly run SPIS are typically met by any computer manufactured after 2005. However, if SPIS is installed on an external server \textbf{make sure that the server has a graphic card with more than 128 Mb of video memory.} If this is missing from the server, the graphical user interface of SPIS cannot be launched, making it impossible to visually track the simulation steps. This will significantly hinder the simulation process, in particular when starting new projects.\par
After downloading and extracting SPIS, move the \verb|SPIS-6.1.0| folder from the Download folder to the preferred application folder.\par
\newpage
To use SPIS, a mesh generator called Gmsh is required. Although Gmsh is a separate software, it is included in the SPIS download package. However, the version of Gmsh included may not be compatible with certain systems, particularly those using a Mac with an M-series chip. In such cases, the macOS (ARM) version of Gmsh should be downloaded directly from the Gmsh website: \url{https://gmsh.info}.

\subsection{Simulation set-up}
\vspace{2mm}

In order to run SPIS simulations, the following steps are required:

\begin{enumerate}[itemsep=-0.2em]
    \item Build a spacecraft geometry, covered by chapter 1.4.
    \item Generate a computational mesh, covered by chapter 1.5.
    \item Define the material library of the spacecraft, covered by chapter 1.7.
    \item Define the circuitry of the spacecraft, covered by chapter 1.8.
    \item Define the spacecraft environment, covered by chapter 1.9.
\end{enumerate}

All these steps will be covered in order, using a simplified spacecraft geometry, a basic cube, to simulate the surface charging of a spacecraft in the ionosphere of Jupiter's moon Ganymede at an altitude of 400 km, as an example.

\subsection{Building the spacecraft 3D model}
\vspace{2mm}

The first step in starting SPIS simulations is to build the spacecraft 3D model and computational mesh. As mentioned earlier, the mesh generator is Gmsh and is included in the SPIS download package. There are three different options for building the spacecraft geometry file, which contains all the information needed to create the computational mesh.

\begin{enumerate}[label=\Alph*., itemsep=-0.2em]
    \item The model can be built in SPIS using the \verb|Edit geometry file| option.
    \item The Gmsh user interface can be used to build the geometry file.
    \item For very simple geometries, the easiest option is often to create a \verb|.geo| file and load it into Gmsh.
\end{enumerate}

This section is a step by step tutorial of option C.\\

A common problem when running SPIS simulations is that the simulation mesh is not properly defined. It is therefore not recommended to build a complex spacecraft model before moving on to steps 2 to 5 of setting up the simulation. Start instead with a very simple spacecraft model, preferably just a cube or cylinder with the relevant proportions, and run a first simulation using that geometry. This approach helps ensure that the rest of the simulation inputs are in order before investing time in developing a more detailed spacecraft model. It supports the creation of a robust simulation set up by isolating potential issues related to the simulation mesh from those associated with the other input parameters, defined in steps 3 to 5 of the simulation set up process.\\

First, create a \verb|.geo| file and open it with any editor of choice (e.g. TextEdit or Notepad). Then, copy, paste, and save the code from the following page, which will create a simple geometry representing a cubic spacecraft inside of a spherical boundary.\\

The following text is a short explanation of the mesh-generating code. Reading through this description is recommended prior to developing a custom mesh. The \verb|OpenCASCADE|  module is used to efficiently define the 3D shapes. The process begins with the definition of mesh sizes for two surfaces, these distinct meshes are automatically connected by lines within the Gmsh software. Since the mesh size of the inner box is smaller than that of the outer sphere, the resulting mesh within the volume of the sphere is non-uniform, increasing in size outward from the box to reach a maximum size of 2 m at the boundary of the sphere. The \verb|Box| object is then created and assigned the ID \verb|1| for reference within the code. Its bottom-left corner is positioned at coordinates $(-1, -1, -1)$, and each side is given a length of 2 m, resulting in a $2\times2\times2$ cube centered at $(0, 0, 0)$. The mesh size of the box is set using the variable \verb|mesh_box|, and the cube is referenced as \verb|Volume{1}| in the corresponding function.\\

Next, the outer spherical boundary is defined using the \verb|Sphere| keyword with an assigned ID of \verb|2|. Its center is set at $(0, 0, 0)$, and the radius is set to 30 m. The mesh size is defined using the variable \verb|mesh_sphere|, analogously to how it was defined for the \verb|Box|. The following line subtracts the volume of the box from that of the sphere, ensuring that the mesh excludes the interior of the spacecraft and is instead restricted to the space between the cube’s exterior and the boundary of the sphere. The box and the sphere are assigned to \verb|Physical Surface(600)| and \verb|Physical Surface(601)|, respectively, for later use in SPIS. The meshed volume is assigned to \verb|Physical Volume(700)|. At this point, the code is complete and ready for visualization in Gmsh. Note: if an error occurs during execution, it is likely due to a missing semicolon at the end of a line.

\begin{minted}{java}
    // Simple spacecraft model 
    // Mika Holmberg 
    // 2025-06-13

    SetFactory("OpenCASCADE");

    // Mesh Sizes
    mesh_box = 0.1;
    mesh_sphere = 2;

    // The spacecraft body 
    Box(1) = {-1, -1, -1, 2, 2, 2};
    MeshSize{ PointsOf { Volume{1}; } } = mesh_box;

    // The outer boundary
    Sphere(2) = {0, 0, 0, 30};
    MeshSize{ PointsOf { Volume{2}; } } = mesh_sphere;

    // Subtract the Box from the Sphere
    BooleanDifference(3) = {Volume{2};Delete;}{Volume{1};Delete;};

    Physical Surface(600) = {1};
    Physical Surface(601) = {2};
    Physical Volume(700) = {3};
\end{minted}

An alternative version of this code that does not use the \verb|OpenCASCADE| module is provided in Appendix A. This version is significantly more complex than the approach presented here. For most geometries, the \verb|OpenCASCADE| module offers clear advantages over the standard method and is therefore the recommended option. Gmsh documentation and tutorials can be found at: \url{https://gmsh.info/doc/texinfo/gmsh.html}.\par

\subsection{The computational mesh}
\vspace{2mm}

The next step is to build the mesh of the computational volume. This can be done in both Gmsh and SPIS. In Gmsh, use the menu on the left hand side of the window, choose \verb|Mesh| and then \verb|3D|. The mesh should then be visible in the computational volume, as shown in Figure \ref{4}.\\

Mesh generation includes an optimisation process that removes low-quality elements that could otherwise lead to computational issues. There are also methods to optimise the mesh even further, such as the \verb|Optimize 3D| or \verb|Optimize 3D (Netgen)|, these methods also reduce the number of mesh elements. Based on experience, the best outcome is usually from using the \verb|Optimize 3D (Netgen)|. Mesh improvements can be monitored using the \verb|Tools| and \verb|Statistics| windows. For example, when creating the mesh shown in Figure \ref{4}, the initial mesh contains 249,626 tetrahedral elements. After applying the \verb|Optimize 3D (Netgen)| once, this number is reduced to 216,805.

\begin{figure}[!ht]
    \centering
    \includegraphics[width=1\textwidth]{fig4.jpg}
    \caption{The mesh of the computational volume.}
    \label{4}
\end{figure}

A useful way to check that the volumes and the mesh are defined correctly is to look at the image of the 3D element faces, as shown in Figure \ref{5}. Open \verb|Tools| and \verb|Options|, choose \verb|Mesh| and remove the markers from \verb|2D element edges| and \verb|3D element edges| and make \verb|3D element faces| visible. Then open \verb|Tools| and \verb|Clipping|, choose \verb|Mesh|. If using 1 for parameter A in Plane 0, the computational mesh will be cut in half, as shown in Figure \ref{5}. Move the object to see the cross section of the mesh.\\

\begin{figure}[!ht]
    \centering
    \includegraphics[width=1\textwidth]{fig5.jpg}
    \caption{The cross-section of the computational volume.}
    \label{5}
\end{figure}

Figure \ref{5} shows that there are no mesh elements present inside the spacecraft, represented by the central square. If the spacecraft volume were not properly closed, mesh elements would also appear inside the spacecraft. Figure \ref{5} also shows that the mesh elements are finer around the spacecraft, defined to be of the order of 0.1 m, and coarser at the outer boundary of the computational volume, where they are defined to be of the order of 2 m.\par
Once the mesh is finalised, it must be saved. Use \verb|File|, \verb|Export|, choose the file format \verb|Mesh - Gmsh MSH (*.msh)| and save the file in the preferred folder and with the preferred name, for example Spacecraft.msh. Click \verb|Save|, choose the \verb|Version 2 ASCII| format and click \verb|OK|. The computational mesh is now ready to be used in SPIS simulations.

\subsection{Starting the SPIS simulation}
\vspace{2mm}

To start SPIS, open a terminal window, go to the SPIS folder and type \verb|sh Spis.sh| or, in Windows, \verb|Spis.bat|. This should open a window like the one shown in Figure \ref{6}. To find the SPIS folder, use the command \verb|ls| to display the contents of the current directory, and use the command \verb|cd| to move to a different folder. For example, typing \verb|ls| shows that the current location is the \verb|mika| directory. The command \verb|cd ../../| is then used to move two directories up to access the \verb|Applications| folder and the SPIS folder named \verb|SPIS-6.1.0-osx64b|. The example is shown in Figure \ref{7}.\\

\begin{figure}[!ht]
    \centering
    \includegraphics[width=1\textwidth]{fig6.jpg}
    \caption{The start of page SPIS.}
    \label{6}
\end{figure}

\begin{figure}[!ht]
    \centering
    \includegraphics[width=0.7\textwidth]{fig7.jpg}
    \caption{How to navigate the SPIS folder.}
    \label{7}
\end{figure}

For Mac users, the first attempt to open SPIS typically results in an error message. This is due to the default security settings of Mac, which only allows apps from App Store or known developers to be opened. To change these settings go to \verb|Security \& Privacy| setting in \verb|System Preferences|. This should open a window like the one shown in Figure \ref{8}. Make sure that \verb|App Store and identified developers| is selected. Next to the blocked app, choose \verb|Allow Anyway| and try to open SPIS again. This process will likely have to be repeated multiple times before the SPIS start window, like the one shown in Figure \ref{6}, will be opened. The main features of the SPIS start window are \verb|Open an existing project|, used to access previously started projects, and \verb|Create a new project|. Select \verb|Create a new project| to begin.\\

\begin{figure}[!ht]
    \centering
    \includegraphics[width=0.7\textwidth]{fig8.jpg}
    \caption{Example of files included in the SPIS download that might have to be manually approved.}
    \label{8}
\end{figure}

\begin{figure}[!ht]
    \centering
    \includegraphics[width=1\textwidth]{fig9.jpg}
    \caption{Name your project and place it in the correct folder.}
    \label{9}
\end{figure}

Clicking \verb|Create a new project| opens the window shown in Figure \ref{9}, where the name of the new project must be specified. As multiple simulations with similar set-ups are likely to be executed, it is recommended to use a naming convention such as Run001. Including the 00 ensures that the projects are sorted in the correct order and not in, for example, the order Run1, Run100, Run2 etc. Select a \verb|Project parent folder| that reflects the simulated environment or specific conditions, for example \verb|Ionosphere|. Since a large number of simulations are typically conducted during the course of a project, it is advisable to create a Readme document to log all the input parameters for each simulation. This will make it much easier to know what separates the different simulation runs. It is not advisable to try to list this in the name of the project since this will lead to excessively long names that still fail to capture all the input parameters in the name.

Since the mesh has already been created using Gmsh, this step can be skipped by selecting the \verb|Skip geometry editor| option. Click \verb|Create project and save| and you have now started your first SPIS project. Congratulations! The next window looks like the one in Figure \ref{10} and this is the point at which the spacecraft model construction begins.

\begin{figure}[!ht]
    \centering
    \includegraphics[width=1\textwidth]{fig10.jpg}
    \caption{The SPIS spacecraft geometry window.}
    \label{10}
\end{figure}

\subsection{The material library} %unclear how to import mesh if already generated, and how to get to figure 20
\vspace{2mm}

The next step is to define the properties of the spacecraft materials using the group editor, as shown in Figure \ref{11}. In the left menu, all defined physical surfaces and volumes will appear. They are identified with the assigned ID numbers. The first physical surface is the spacecraft with the ID number \verb|600|. Selecting this surface allows the \verb|Group type| to be set, which should be \verb|Spacecraft surface group|. This group type includes a range of properties; most of them can be left at their default values, but the most important one is the \verb|S/C material|. SPIS already has a material library with many common spacecraft materials; for this example, the \verb|Kapton Black (2K) material properties| is selected, which is a common spacecraft surface material.\\

\begin{figure}[!ht]
    \centering
    \includegraphics[width=1\textwidth]{fig11.jpg}
    \caption{The group editor where the material properties are defined.}
     \label{11}
\end{figure}

\begin{figure}[!ht]
    \centering
    \includegraphics[width=1\textwidth]{fig12.jpg}
    \caption{Choose the spacecraft surface material.}
    \label{12}
\end{figure}

\begin{figure}[!ht]
    \centering
    \includegraphics[width=1\textwidth]{fig13.jpg}
    \caption{The material properties are listed in the left menu.}
    \label{13}
\end{figure}

Select the surface material and click \verb|OK|. The different material properties will be added to this specific surface, as shown in Figure \ref{12}. When \verb|Kapton Black (2K) material properties| is selected and expanded, a list of associated properties appears, as shown in Figure \ref{13}. These properties can be modified if needed. The definition of the different properties are listed in Appendix B. For example, to explore how the spacecraft charging will change with a less conductive version of black kapton, the bulk conductivity parameter called BUC can be adjusted. For black kapton this value is set to -1, which means that it is simulated as a perfect conductor. To test an alternative conductivity value, click on the parameter \verb|BUC| and change the value in the right window, as shown in Figure \ref{14}.\\


\begin{figure}[!ht]
    \centering
    \includegraphics[width=1\textwidth]{fig14.jpg}
    \caption{To change a material property.}
    \label{14}
\end{figure}

The next group is the external boundary with ID 601. Click on \verb|FaceGroup - 601|,  choose \verb|External boundary group| and \verb|OK|. Then click on \verb|VolumeGroup -700|, choose \verb|Computational volume group| and \verb|OK|. The group editor should then look like in Figure \ref{15}. Then click \verb|Next|.

\begin{figure}[!ht]
    \centering
    \includegraphics[width=1\textwidth]{fig15.jpg}
    \caption{The group editor after all groups have been defined.}
    \label{15}
\end{figure}

\subsection{The circuitry}
\vspace{2mm}

The next step is to define the electrical circuit. This is where to specify how the various parts of the spacecraft are electrically connected. The different parts can be connected using resistors (called R), capacitors (C), or voltage generators (V). An example would be\\

V  0  1  -10\\
R  0  2  1.e6\\

which means that a biased voltage of -10 V would be imposed between surfaces 0 and 1, and a resistor of 1.e6 Ohm is set between surfaces 0 and 2. Since we only have one spacecraft surface, we can leave this section blank, as shown in Figure \ref{16}, and just press \verb|Next|. \\

\begin{figure}[!ht]
    \centering
    \includegraphics[width=1\textwidth]{fig16.jpg}
    \caption{The electric circuit editor.}
    \label{16}
\end{figure}

\subsection{The environment}
\vspace{2mm}

The final step before starting the simulations is to define the environment in which the spacecraft is located. When issues arise during simulations, they are most often caused by poorly defined spacecraft geometry or mesh, or a badly defined environment. It is therefore especially important to have a solid understanding of the space environment of the spacecraft. Take the time to carefully review and double check all environmental parameters. If you are not an expert yourself, consult an expert to ensure that the parameters accurately represent the environment.\\

The window used to define the environmental parameters is named \verb|Global parameters| and looks like the window shown in Figure \ref{17}.\\

\begin{figure}[!ht]
    \centering
    \includegraphics[width=1\textwidth]{fig17.jpg}
    \caption{Here you define your spacecraft environment.}
     \label{17}
\end{figure}

If more parameters are displayed than those shown in Figure \ref{17}, then change the \verb|Expertise| \verb|level| to \verb|LOW|. These are the initial parameters to be configured. The different options are \verb|Plasma|, \verb|Surface Interaction|, \verb|Transitions| etc. Begin by defining the plasma environment. Set the electron and ion densities, \verb|electronDensity| and \verb|ionDensity|, which are given in m$^{-3}$. By default, two different electron populations and two different ion populations can be defined. For example, let's use the densities found in the ionosphere of Jupiter's moon Ganymede at an altitude of around 400 km. Here the electron and ion density is around 100 cm$^{-3}$. It is important to ensure that the environment is neutral, so that the total electron density equals the total ion density. Almost all natural plasma environments are quasineutral, which for the purposes of these simulations can be treated as fully neutral.\\

Electron and ion temperatures and velocities can also be specified. For the Ganymede ionosphere, typical values include an electron temperature of 2 eV and an ion temperature of 50 eV. The velocity here will mainly be due to the movement of the spacecraft, so set the plasma velocities to 0. The ion species present in the plasma can also be selected, hydrogen H+ ions are used in this example. In addition, the number of particle pushers called \verb|pusherThreadNb| can be selected, that is the number of available processor cores, and the average number of super-particles per cell called \verb|avPartNbPerCell|, which will determine how noisy the simulations are. A larger number is better, but it is also computationally heavier, so start with a low number, like 5 and increase it for more detailed simulations once the simulation runs smoothly.\\

Proceeding to the \verb|Surface Interaction| opens a window as shown in Figure \ref{18}. Here, the location of the Sun must be set, where the shown example has the Sun at a distance of 1 AU, from the spacecraft, on the positive z axis. For a spacecraft located in Ganymede's ionosphere, the solar flux must be scaled to match conditions at Jupiter's orbit. The average distance between Jupiter and the Sun is 5.2 AU, hence the Sun's location should be set to $1/5.2^2 = 0.037$. The position of the Sun can be reviewed by selecting \verb|Sun orientation|, this will open up a window like the one shown in Figure \ref{19}. 

\begin{figure}[!ht]
    \centering
    \includegraphics[width=1\textwidth]{fig18.jpg}
    \caption{Where surface interactions are defined.}
    \label{18}
\end{figure}

\begin{figure}[!ht]
    \centering
    \includegraphics[width=1\textwidth]{fig19.jpg}
    \caption{The Sun orientaiton viewer.}
    \label{19}
\end{figure}

Please note that the location of the Sun and the spacecraft in this visualisation is, of course, not to scale.\\

Another option in the surface interaction window is the \verb|photoEmission|. This parameter determines whether to include or exclude the emission of photoelectrons from the spacecraft. These are electrons that are emitted from the spacecraft surface due to the interaction with photons from the Sun (or any other photon source). If the spacecraft is in eclipse, this parameter should be set to 0, if the spacecraft is in sunlight the parameters should stay at 3.\\

For our simple example, default values are used for the following list of parameters, called \verb|Transitions|, \verb|Spacecraft|, \verb|Outputs|, \verb|Poisson equation|, \verb|Volume Interactions|, and \verb|Scenario|. The list after \verb|Volume Interactions| defines the values of the magnetic field that the spacecraft is located in. Clicking on \verb|B Field| will open a window like the one shown in Figure \ref{20}.\\

Please be aware that the magnetic field should be given in Tesla. For our example, we use a magnetic field strength of 500 nT in the z direction.\\

\begin{figure}[!ht]
    \centering
    \includegraphics[width=0.96\textwidth]{fig20.jpg}
    \caption{Where to set the magnetic field values.}
    \label{20}
\end{figure}

The final step in the set up involves specifying the simulation \verb|duration| and the simulation time step called \verb|simulationDt|. The time step must be shorter than the duration; it is recommended that \verb|simulationDt| is at least $<$ \verb|duration|/10. For our example, we use a \verb|duration| of 1 s and a \verb|simulationDt| of 0.05 s. %where?\par
After setting the environmental parameters it is time to start the simulation by selecting \verb|Finalize run configuration and save| \verb|project|. This opens the simulation  launch window shown in Figure \ref{21}. To start the simulation press \verb|Launch simulation|. When the simulation is running, the time steps should be continuously updated in the \verb|Log console| and some parameters, like \verb|Individual currents on spacecraft|, should be shown in the window above the \verb|Log| \verb|console|, as shown in Figure \ref{22}.

\begin{figure}[!ht]
    \centering
    \includegraphics[width=1\textwidth]{fig21.jpg}
    \caption{Where to launch the simulation.}
    \label{21}
\end{figure}

\begin{figure}[!ht]
    \centering
    \includegraphics[width=1\textwidth]{fig22.jpg}
    \caption{The simulation is running.}
    \label{22}
\end{figure}

\clearpage
\section{SPIS source code primer}

SPIS has a Graphical User Interface (GUI) that allows the user to perform simulations between a spacecraft and its plasma environment using the mouse cursor. The underlying computations and physical theory of these simulations are complex and not exposed through the GUI. To explore the mathematical formulations and algorithms that drive these simulations, it is necessary to look at the SPIS Java source code, which can be accessed from the SPIS Git repositories: \url{https://www.spis.org/faq/how-to-compile-spis/}. A registered SPIS account is required to access these repositories. At the time of writing, the source code is distributed across four repositories: \verb|SPIS-Instruments|, \verb|SPIS-NUM|, \verb|SPIS-NUM Plugins| and \verb|SPIS-UI|. The primary simulation code is in the \verb|SPIS-NUM| repository. This is complemented by \verb|SPIS-NUM Plugins| and, where relevant, by \verb|SPIS-Instruments|.\\

The code itself is written in the object-oriented Java programming language and follows the Maven structure, which is a way of formatting the file structure of Java projects. The recommended method to open this project is to use the Eclipse software (\url{https://eclipseide.org/}), or alternatively Visual Studio Code with the official Java extension from Microsoft (\url{https://code.visualstudio.com/download}). These tools provide efficient code navigation through the embedded documentation and clickable access to class and method definitions. Once the repository is opened, the relevant source code can be found under the directory path \verb|/src/main/java/spis/|. Below is a short summary of what the folders in this directory contains:

\begin{itemize}
    \item \textbf{Top:} Contains the simulation object \verb|Simulation| as well as the spacecraft \verb|SC| and plasma \verb|Plasma| objects. This allows the computation of the plasma-material interactions on the spacecraft, plasma dynamics, and integration of spacecraft potentials with respect to time. The major fields acting on the spacecraft are defined in \verb|Circuit|, \verb|SurfMesh|, and \verb|SurfField|, while those of the plasma contain the volume mesh, electromagnetic fields, and particle populations. Also contains default objects and values for the simulation, transition behaviour and \verb|Top| where the main program is run.
    \item \textbf{Vol:} Contains all of the necessary mathematical background to calculate volume distributions, fields, interactions, and meshes in 3D space.
    \item \textbf{Surf:} Contains all of the necessary mathematical background to calculate surface distributions, fields, interactions, and meshes on a 2D surface. Also contains the handling of spacecraft materials.
    \item \textbf{Circ:} Contains all of the necessary mathematical background to define and calculate the spacecraft's representation as an electrical circuit.
    \item \textbf{Solver:} Contains all of the necessary models to solve the equations defined in the \verb|Vol|, \verb|Surf|, and \verb|Circ| folders.
    \item \textbf{Util:} Contains diverse programming and mathematical utilities.
\end{itemize}

This section provides only a brief overview of the code's capabilities and is intended to provide initial guidance. For those planning to explore the source code in greater depth, consulting the official documentation is strongly recommended. The documentation is included with the SPIS installation, located in the \verb|documentation| folder. The relevant material for source code exploration can be found in \verb|DocSpisNum| where the \verb|API| sub-folder is focused on Java class definitions, while the \verb|HowTo| sub-folder provides a broader, more conceptual overview, serving as the recommended starting point. Opening the \verb|index.html| file in either sub-folder allows local navigation of the documentation through a web browser. This section is based on the file: \verb|/documentation/DocSpisNum/HowTo/| \verb|NUM architecture.html|. For a more general approach, a user manual is also available in the \verb|DocSpisNum| folder.

\newpage
\section{Appendix}
\subsection{Appendix A}

This appendix presents an alternative way to build the spacecraft geometry model presented in Chapter 1.4.\\

We will begin by building a simple cube that represents the spacecraft, as shown in Figure \ref{A1}. The parameters (l and b) that are defined in the beginning of the file gives the resolution of the mesh at the surface of the spacecraft, here called l, and at the outer boundary of the mesh, here called b.\par

\begin{figure}[!ht]
    \centering
    \includegraphics[width=1\textwidth]{figA1.jpg}
    \caption{The beginning of the file that defines the spacecraft geometry.}
    \label{A1}
\end{figure}

The Points give the coordinates for the points that define the geometry and also the mesh resolution at that specific point. For example, \mint{java}|Point(100) = {-1, -1, -1, l};| states that a point with ID number \verb|100| should be located at x = -1, y = -1, z = -1 and the mesh connected to this point should have mesh resolution \verb|l|, which is given a length of 0.1 m in the lines above. Points \verb|100| to \verb|107| give the corner points of the sphere. We also have to define the outer boundary of our computational volume, these are given by Points \verb|108| to \verb|114|. Please be careful to use a large enough simulation geometry around the spacecraft, if the simulation volume is too small this will provide erroneous results. Large enough means that the computational volume should include any changes to the local environment caused by the presence of the spacecraft, this size will depend on the environment and spacecraft design. In this example we start with a volume with a diameter of 60 m for a cube shaped spacecraft with sides of 2 m.\\

After defining the points, define lines that connect the points, as shown in Figure \ref{A2}. Here the three most commonly used options are

\begin{enumerate}[itemsep=-0.2em]
    \item Line, which is only used for straight lines.
    \item Circle, which is only used for sections of circles.
    \item Ellipse, which is used for sections of ellipses.
\end{enumerate}

\begin{figure}[!ht]
    \centering
    \includegraphics[width=1\textwidth]{figA2.jpg}
    \caption{The lines that connects the points of the spacecraft geometry.}
    \label{A2}
\end{figure}

For example, \mint{java}|Line(200) = {100, 101};| states that a line with ID number \verb|200| should be located between the points with ID numbers \verb|100| and \verb|101|. The direction of the line does not matter in this instance, so \mint{java}|Line(200) = {100, 101}; and Line(200) = {101, 100};| will give the same result.\\

In Gmsh, the circle sections must be smaller than 180 degrees to ensure that the software correctly identifies the corresponding arc. The first point defines the start of the arc, the second defines the center of the circle and the third defines the end of the arc. After defining the lines, each surface must be defined, as shown in Figure \ref{A3}. It is important to distinguish between surfaces with different materials, since the second step of the simulation set up is to define the materials and their properties. At that stage, each surface material must be defined individually. To define a surface, begin by listing the lines that form the boundary of that surface. For instance, \mint{java}|Line Loop(300) = {200, 201, 202, 203};| states that the lines with ID numbers \verb|200|, \verb|201|, \verb|202|, and \verb|203| form a closed line loop with ID number \verb|300|. This closed line loop defines a plane surface so use \mint{java}|Plane Surface (301) = {300};| to state that the surface with ID \verb|301| is the surface within the Line Loop with ID \verb|300|.

\begin{figure}[!ht]
    \centering
    \includegraphics[width=1\textwidth]{figA3.jpg}
    \caption{How to define the surfaces of the spacecraft model.}
    \label{A3}
\end{figure}

The directions of the enclosed lines that make up a surface are crucial. For example, Line Loop \verb|312| is made up of the line segments \verb|212|, \verb|221| and \verb|216|, but the directions of the lines are not all the same. Line \verb|212| is defined from Point \verb|109| to \verb|110|, Line \verb|221| is defined from Point \verb|113| to \verb|110|, and Line \verb|216| is defined from Point \verb|109| to \verb|113|. Hence, by reversing the direction of lines \verb|221| and \verb|216| a closed loop is formed with all segments oriented in the same direction. The order of the points will then be \verb|109|, \verb|110|, \verb|113| and back to \verb|109|. Alternatively, the line loop can be defined in the opposite direction as \mint{java}|{-212, 221, 216}| that would achieve the same result. Since it is easy to make mistakes at this step, it is imperative to double check wether the line loops are correctly defined before moving on. This can be done by opening the geometry file in Gmsh. Click on the Gmsh icon in the Application folder and choose \verb|File| and \verb|Open| and choose the geometry file. For the simple geometry that we are using here, this opens a window like the one in Figure 7. If the Gmsh version included in the downloaded SPIS package is being used, Gmsh can be found within the SPIS folder in \verb|dependencies/thirdparty|.\\

The blue circle shown in Figure \ref{A4} is the outer boundary of the computational volume and the small square represents the spacecraft. If one of the line loops has been defined incorrectly, the window would look like in Figure \ref{A5}. Please note the red error message displayed at the bottom of the window.

\begin{figure}[!ht]
    \centering
    \includegraphics[width=1\textwidth]{figA4.jpg}
    \caption{The spacecraft geo file opened in Gmsh.}
    \label{A4}
\end{figure}

\begin{figure}[!ht]
    \centering
    \includegraphics[width=0.99\textwidth]{figA5.jpg}
    \caption{The red text shows the error message of Gmsh.}
    \label{A5}
\end{figure}

\begin{figure}[!ht]
    \centering
    \includegraphics[width=0.99\textwidth]{figA6.jpg}
    \caption{The geometry can be rotated and enlarged.}
    \label{A6}
\end{figure}

\begin{figure}[!ht]
    \centering
    \includegraphics[width=1\textwidth]{figA7.jpg}
    \caption{The surfaces are crossed by grey dashed lines.}
    \label{A7}
\end{figure}

The error message shows that something has been defined incorrectly. Clicking on the red line reveals the full error message: \textcolor{red}{Curve loop 312 is wrong, Wrong definition of surface 313: 1 borders instead of 3 or 4}. Rotating the geometry and zooming in allows for visual inspection to ensure that all the points and lines are properly defined, as can be seen in Figure \ref{A6}.\\

To verify that all the surfaces are correctly defined press \verb|Tools|, \verb|Options|, \verb|Geometry| and mark \verb|Surfaces|. This will include grey dashed lines crossing all the defined surfaces, as shown in Figure \ref{A7}. Enabling the \verb|Volumes| option reveals a yellow sphere at the center of the computational volume. Defining the computational volume begins with creating a Surface Loop, as shown in Figure \ref{A8}. For example, \mint{java}|Surface Loop(400) = {301, 303, 305, 307, 309, 311};| states that a surfaces loop with ID number \verb|400| contains all the plane surfaces of the sphere that represents the spacecraft. Surface Loop \verb|401| contains all the surfaces of the outer boundary. To define the computational volume use \mint{java}|Volume(500) = {401, 400};| where the inner surface loop \verb|400| is subtracted from the outer surface loop \verb|401|, resulting in a volume between the spacecraft model and the outer boundary, which is the computational volume. It is important to make sure that all volumes are closed volumes and that no surfaces are missing.\\

After defining the surfaces and volumes, they have to be grouped together in Gmsh into what is called physical surfaces and volumes. An example is shown in Figure \ref{A8}. The Physical Volume should be the same computational volume as defined earlier.\newpage

\begin{figure}[!ht]
    \centering
    \includegraphics[width=1\textwidth]{figA8.jpg}
    \caption{Example of how to define the volumes and physical groups.}
    \label{A8}
\end{figure}


Code:

\begin{minted}{java}

    // Simple spacecraft model
    // Mika Holmberg
    // 2024-07-31

    // Lengths

    l = 0.1;	// Characteristic length for mesh generation
    b = 2;		// Characteristic length for mesh generation on the boundary

    // The spacecraft body

    Point(100) = {-1, -1, -1, l};
    Point(101) = {-1, -1, 1, l};
    Point(102) = {-1, 1, 1, l};
    Point(103) = {-1, 1, -1, l};
    Point(104) = {1, -1, -1, l};
    Point(105) = {1, -1, 1, l};
    Point(106) = {1, 1, 1, l};
    Point(107) = {1, 1, -1, l};

    // The outer boundary

    Point(108) = {0, 0, 0, b};
    Point(109) = {30, 0, 0, b};
    Point(110) = {0, 30, 0, b};
    Point(111) = {-30, 0, 0, b};
    Point(112) = {0, -30, 0, b};
    Point(113) = {0, 0, 30, b};
    Point(114) = {0, 0, -30, b};

    // Lines for the spacecraft

    Line(200) = {100, 101};
    Line(201) = {101, 102};
    Line(202) = {102, 103};
    Line(203) = {103, 100};
    Line(204) = {104, 105};
    Line(205) = {105, 106};
    Line(206) = {106, 107};
    Line(207) = {107, 104};
    Line(208) = {105, 101};
    Line(209) = {102, 106};
    Line(210) = {103, 107};
    Line(211) = {104, 100};

    // Lines for the outer boundary

    Circle(212) = {109, 108, 110};
    Circle(213) = {110, 108, 111};
    Circle(214) = {111, 108, 112};
    Circle(215) = {112, 108, 109};
    Circle(216) = {109, 108, 113};
    Circle(217) = {113, 108, 111};
    Circle(218) = {111, 108, 114};
    Circle(219) = {114, 108, 109};
    Circle(220) = {112, 108, 113};
    Circle(221) = {113, 108, 110};
    Circle(222) = {110, 108, 114};
    Circle(223) = {114, 108, 112};

    // Defining the surfaces on the satellite

    Line Loop(300) = {200, 201, 202, 203};
    Plane Surface(301) = {300};
    Line Loop(302) = {204, 205, 206, 207};
    Plane Surface(303) = {302};
    Line Loop(304) = {204, 208, -200, -211};
    Plane Surface(305) = {304};
    Line Loop(306) = {202, 210, -206, -209};
    Plane Surface(307) = {306};
    Line Loop(308) = {205, -209, -201, -208};
    Plane Surface(309) = {308};
    Line Loop(310) = {211, -203, 210, 207};
    Plane Surface(311) = {310};

    // Defining the surfaces of the outer boundary

    Line Loop(312) = {212, -221, -216};
    Surface(313) = {312};
    Line Loop(314) = {213, -217, 221};
    Surface(315) = {314};
    Line Loop(316) = {214, 220, 217};
    Surface(317) = {316};
    Line Loop(318) = {215, 216, -220};
    Surface(319) = {318};
    Line Loop(320) = {212, 222, 219};
    Surface(321) = {320};
    Line Loop(322) = {213, 218, -222};
    Surface(323) = {322};
    Line Loop(324) = {214, -223, -218};
    Surface(325) = {324};
    Line Loop(326) = {215, -219, 223};
    Surface(327) = {326};

    // Defining the computational volume

    Surface Loop(400) = {301, 303, 305, 307, 309, 311}; // Spacecraft
    Surface Loop(401) = {313, 315, 317, 319, 321, 323, 325, 327}; // Outer boundary
    Volume(500) = {401, 400};

    // Physical groups

    Physical Surface(600) = {301, 303, 305, 307, 309, 311}; // Spacecraft
    Physical Surface(601) = {313, 315, 317, 319, 321, 323, 325, 327}; // Outer boundary

    Physical Volume(700) = {500};

\end{minted}

\subsection{Appendix B}

The table below lists the Z93C55 material properties and gives a description of the different material properties that have to be listed in the material library of a SPIS simulation.

\bgroup
\def\arraystretch{1.5}

\begin{center}
\begin{tabular}{|c|c|c|p{9cm}|}
    \hline
    \textbf{Value} & \textbf{Name} & \textbf{Unit} & \textbf{Description} \\
    \hline
    0.3 & RDC & - & Relative dielectric constant \\
    \hline
    $10^{-13}$ & BUC & 1/($\si{ohm.m})$ & Bulk conductivity\\
    \hline
    5 & ATN & - & Atomic number \\ 
    \hline
    2.2 & MSEY & - & Maximum Secondary Electron Emission (SEE) yield for electron impact \\
    \hline
    0.3 & PEE & keV & Primary electron energy that produces maximum SEE yield \\ 
    \hline
    0.5 & RPN1 & Angstrom & Range parameter $r_1$ in the range expression $r_1(E/E_0)^{n_1}+r_2(E/E_0)^{n_2}$, with $E_0=1$ keV (or equivalently with no $E_0$ coefficient and $E$ in keV) \\ 
    \hline
    50 & RPR1 & - & Range parameter $n_1$ \\ 
    \hline
    1.55 & RPN2 & Angstrom & Range parameter $r_1$ \\ 
    \hline
    100 & RPR2 & - & Range parameter $n_2$ \\ 
    \hline
    0.244 & SEY & - & Secondary electron yield due to impact of 1 keV protons \\ 
    \hline
    230 & IPE & keV & Incident proton energy that produces maximum seconday electron yield \\ 
    \hline
    $2 \times 10^{-5}$ & PEY & A/m$^{2}$ at 1 AU & Photoelectron current for normally incident sunlight \\ 
    \hline
    $1.3 \times 10^{21}$ & SRE & ohm & Surface resistivity \\ 
    \hline
    $10^4$ & MAP & V & Maximum (absolute) potential attainable before a discharge occurs \\ 
    \hline
    2$\times 10^{3}$ & MPD & V & Maximum potential difference between surface and underlying conductor before a discharge occurs \\ 
    \hline
    $1.4 \times 10^{-13}$ & RCC & 1/($\si{ohm.m})$ & Radiation induced conductivity coefficient $K$ in the law $K(rate/rate_0)^D$ with $rate_0$=1 $\si{rad/s}$ \\ 
    \hline
\end{tabular}
\end{center}
\egroup

\end{document}